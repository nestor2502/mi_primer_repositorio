\documentclass{article}

\usepackage[spanish]{babel}
\usepackage[utf8]{inputenc}
\usepackage[T1]{fontenc}

\usepackage[
  top=3cm,
  bottom=3cm,
  left=2cm,
  right=2cm,
  heightrounded,
]{geometry}
\usepackage{enumerate}
\author{Nombre Del Autor}
\title{Mi título}
\date{\today}

\begin{document}

\maketitle

Plantilla de \LaTeX.

% https://www.lipsum.com
\section{Ejemplos de enumeracio en Latex}
\subtitle{nestor y ari}
\begin{enumerate}
\item Este es el primer parrafo en el primer nivel
  \begin{enumerate}
  \item Queremos ver que pasa al anidar en el segundo nivel
  \end{enumerate}
  
    

  
    \begin{enumerate}
      
      
      
\item Este es otro ejemplo de enumeración
\item Aquí hay otro renglon
    \end{enumerate}
    

    \begin{enumerate}[I.]
    \item Ejemplo de inciso con numeros romanos
    \item Segundo inciso con numeros romanos
    \end{enumerate}

  
    Este es un ejemplo de una formula matematica \\
    $x_i + y_i=0$. Podemos seguir escribiendo\\
    Ejemplo de una fraccion $\frac{x}{y}$\\
    Ejemplo de raiz $\sqrt{x}$\\
    Ejemplo de signo especial $\alpha$
    La chicharronera:\\
    $$\frac{-\beta\pm\sqrt{\beta^2-4\alpha\gamma}}{2\alpha}$$
    \newline
    hello jejeje
\end{document}











































































    
    




\end{document}
